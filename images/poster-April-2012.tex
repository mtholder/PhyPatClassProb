% --------------------------------------------------------------------------- %
% Poster for the ECCS 2011 Conference about Elementary Dynamic Networks.      %
% --------------------------------------------------------------------------- %
% Created with Brian Amberg's LaTeX Poster Template. Please refer for the     %
% attached README.md file for the details how to compile with `pdflatex`.     %
% --------------------------------------------------------------------------- %
% $LastChangedDate:: 2011-09-11 10:57:12 +0200 (V, 11 szept. 2011)          $ %
% $LastChangedRevision:: 128                                                $ %
% $LastChangedBy:: rlegendi                                                 $ %
% $Id:: poster.tex 128 2011-09-11 08:57:12Z rlegendi                        $ %
% --------------------------------------------------------------------------- %
\documentclass[a0paper,landscape]{baposter}

\usepackage{relsize}		% For \smaller
\usepackage{url}			% For \url
\usepackage{epstopdf}	% Included EPS files automatically converted to PDF to include with pdflatex
\usepackage{natbib}

%%% Global Settings %%%%%%%%%%%%%%%%%%%%%%%%%%%%%%%%%%%%%%%%%%%%%%%%%%%%%%%%%%%

\graphicspath{{./}}	% Root directory of the pictures 
\tracingstats=2			% Enabled LaTeX logging with conditionals

%%% Color Definitions %%%%%%%%%%%%%%%%%%%%%%%%%%%%%%%%%%%%%%%%%%%%%%%%%%%%%%%%%

\definecolor{bordercol}{RGB}{40,40,40}
\definecolor{headercol1}{RGB}{186,215,230}
\definecolor{headercol2}{RGB}{0,60,255}
\definecolor{headerfontcol}{RGB}{0,0,0}
\definecolor{boxcolor}{RGB}{186,215,230}

%%%%%%%%%%%%%%%%%%%%%%%%%%%%%%%%%%%%%%%%%%%%%%%%%%%%%%%%%%%%%%%%%%%%%%%%%%%%%%%%
%%% Utility functions %%%%%%%%%%%%%%%%%%%%%%%%%%%%%%%%%%%%%%%%%%%%%%%%%%%%%%%%%%

%%% Save space in lists. Use this after the opening of the list %%%%%%%%%%%%%%%%
\newcommand{\compresslist}{
	\setlength{\itemsep}{1pt}
	\setlength{\parskip}{0pt}
	\setlength{\parsep}{0pt}
}

%%%%%%%%%%%%%%%%%%%%%%%%%%%%%%%%%%%%%%%%%%%%%%%%%%%%%%%%%%%%%%%%%%%%%%%%%%%%%%%
%%% Document Start %%%%%%%%%%%%%%%%%%%%%%%%%%%%%%%%%%%%%%%%%%%%%%%%%%%%%%%%%%%%
%%%%%%%%%%%%%%%%%%%%%%%%%%%%%%%%%%%%%%%%%%%%%%%%%%%%%%%%%%%%%%%%%%%%%%%%%%%%%%%

\begin{document}
\typeout{Poster rendering started}

%%% Setting Background Image %%%%%%%%%%%%%%%%%%%%%%%%%%%%%%%%%%%%%%%%%%%%%%%%%%
\background{
	\begin{tikzpicture}[remember picture,overlay]%
	\draw (current page.north west)+(-2em,2em) node[anchor=north west]
	{\includegraphics[height=1.1\textheight]{background}};
	\end{tikzpicture}
}

%%% General Poster Settings %%%%%%%%%%%%%%%%%%%%%%%%%%%%%%%%%%%%%%%%%%%%%%%%%%%
%%%%%% Eye Catcher, Title, Authors and University Images %%%%%%%%%%%%%%%%%%%%%%
\begin{poster}{
	grid=false,
	% Option is left on true though the eyecatcher is not used. The reason is
	% that we have a bit nicer looking title and author formatting in the headercol
	% this way
	eyecatcher=true, 
	borderColor=bordercol,
	headerColorOne=headercol1,
	headerColorTwo=headercol2,
	headerFontColor=headerfontcol,
	% Only simple background color used, no shading, so boxColorTwo isn't necessary
	boxColorOne=boxcolor,
	headershape=roundedright,
	headerfont=\Large\sf\bf,
	textborder=rectangle,
	background=user,
	headerborder=open,
  boxshade=plain
}
%%% Eye Cacther %%%%%%%%%%%%%%%%%%%%%%%%%%%%%%%%%%%%%%%%%%%%%%%%%%%%%%%%%%%%%%%
{
	Eye Catcher, empty if option eyecatcher=false - unused
}
%%% Title %%%%%%%%%%%%%%%%%%%%%%%%%%%%%%%%%%%%%%%%%%%%%%%%%%%%%%%%%%%%%%%%%%%%%
{\bf
	Algorithms for Calculating Pattern Class Probabilities on Phylogenetic Trees
}
%%% Authors %%%%%%%%%%%%%%%%%%%%%%%%%%%%%%%%%%%%%%%%%%%%%%%%%%%%%%%%%%%%%%%%%%%
{
	\vspace{1em} Jordan M. Koch and Mark T. Holder\\
	{\smaller \url{j772k779@ku.edu}, \url{mtholder@ku.edu}}
}
%%% Logo %%%%%%%%%%%%%%%%%%%%%%%%%%%%%%%%%%%%%%%%%%%%%%%%%%%%%%%%%%%%%%%%%%%%%%
{
% The logos are compressed a bit into a simple box to make them smaller on the result
% (Wasn't able to find any bigger of them.)

\setlength\fboxsep{0pt}
\setlength\fboxrule{0.5pt}
	\fbox{
		\begin{minipage}{4em}
			%\includegraphics[width=10em,height=4em]{colbud_logo}
			%\includegraphics[width=4em,height=4em]{elte_logo} \\
			%\includegraphics[width=10em,height=4em]{dynanets_logo}
			%\includegraphics[width=4em,height=4em]{aitia_logo}
		\end{minipage}
	}
}

\headerbox{Introduction}{name=intro,column=0,row=0}{
A wide variety of evolutionary analyses are based upon the coupling of phylogenetic trees with models of how biological traits change during evolution. 
Felsenstein's \citep{Felsenstein1981} pruning algorithm makes it feasible to calculate the probability of any particular pattern of data arising on a phylogeny. 
In some contexts, one needs to calculate the probability that any member of a broad class of patterns will arise on the tree. 
For example, the model adequacy approach of Waddell et al. (2009) requires calculating the probability of several classes of patterns.  
Extending the morphological models of Lewis \citep{Lewis2001} to deal with many data sets requires calculating the probability of any parsimony-informative pattern arising (`parsimony' referring to the simplest explanation of the data, and parsimony-informative referring to those patterns which affect phylogenetic estimation). Application of the approaches of Waddell {\em et al.~}\citep{WaddellOP2009} and Lewis\citep{Lewis2001} are limited because the only known methods for calculating the probability of a class of patterns involve either simulating a large amount of data or exhaustively considering every member of the class. Neither approach is feasible on large trees.

We are developing dynamic programming algorithms to calculate the probabilities of pattern classes in one pass down a phylogenetic tree. The algorithms include a general approach (applicable to any standard model of character evolution) as well as optimizations for the fully symmetric models (e.g. those of Lewis\citep{Lewis2001}). We are implementing these algorithms in open source software written in C++, and plan to include the approaches in the GARLI \citep{GARLI} software package for use in inferring evolutionary trees.
%\includegraphics[width=\linewidth]{time_windows}
}

\headerbox{Research Goals}{name=research goals,column=0,below=intro}{
We are working to discover algorithms to implement the specialization for symmetric models.  Recognizing these symmetric cases will simplify the code's representation in C++, decreasing redundancy and increasing efficiency.  
}

\headerbox{Results}{name=results,span=2,column=1,row=0}{
}
\headerbox{Results2}{name=results2,column=1,below=results}{
}
\headerbox{Results4}{name=results4,column=2,below=results}{
}
\headerbox{Results3}{name=results3,column=3,row=0}{
}

\headerbox{Future Work}{name=future work,column=2,row=3}{
We plan to include our approaches in the GARLI \citep{GARLI} software package to be used in inferring evolutionary trees.  We also plan on applying the model adequacy approach of Waddell {\em et al.~}\citep{WaddellOP2009}, as well as Felsenstein's \citep{Felsenstein1981} pruning algorithm to larger data sets, in order to have a better idea of certain evolutionary behaviors.
}

\headerbox{Acknowledgements}{name=acknowledgements,column=3,below=results3}{
\smaller						% Make the whole text smaller
\vspace{-0.4em}			% Save some space at the beginning
The authors would like to thank NSF and the IMSD program for funding.
} 

\headerbox{References}{name=references,column=3,below=acknowledgements}{
\smaller													% Make the whole text smaller
\vspace{-0.4em} 										% Save some space at the beginning
\bibliographystyle{plain}							% Use plain style
\renewcommand{\section}[2]{\vskip 0.05em}		% Omit "References" title
\bibliography{../pattern_class}
}



\end{poster}
\end{document}
