\documentclass[11pt]{article}
\usepackage{graphicx}
\DeclareGraphicsRule{.tif}{png}{.png}{`convert #1 `dirname #1`/`basename #1 .tif`.png}
\usepackage{wrapfig}
\textwidth = 6.5 in
\textheight = 9 in
\oddsidemargin = 0.0 in
\evensidemargin = 0.0 in
\topmargin = 0.0 in
\headheight = 0.0 in
\headsep = 0.0 in
\parskip = 0.2in
\parindent = 0.0in
\usepackage{setspace}
\singlespacing

\usepackage{titlesec}
\usepackage{paralist} %compactenum

%\newtheorem{theorem}{Theorem}
%\newtheorem{corollary}[theorem]{Corollary}
%\newtheorem{definition}{Definition}
\usepackage{tipa}
\usepackage{bm}
\usepackage{amsfonts}
\usepackage[mathscr]{eucal}

\titlespacing*{\section}{0cm}{1mm}{-3mm}
\titlespacing*{\subsection}{0cm}{-3mm}{-3mm}
% Use the natbib package for the bibliography
\usepackage[round]{natbib}
\bibliographystyle{apalike} 
\newcommand{\exampleMacro}[1]{\mu_{#1}}


\renewcommand{\subsubsection}[1]{%
\addcontentsline{toc}{subsubsection}{#1}
\noindent\textbf{#1}:}


\usepackage{xspace}
%\usepackage{ulem}
\usepackage{url}
%\usepackage{ensuremath}
\usepackage{hyperref}


\hypersetup{backref,  pdfpagemode=FullScreen,  linkcolor=blue, citecolor=red, colorlinks=true, hyperindex=true}


\newcommand{\treeRoot}{\ensuremath{\rho}\xspace}
\newcommand{\parsLength}{\ensuremath{\ell}\xspace}
\newcommand{\numObserved}{\ensuremath{s}\xspace}
\newcommand{\inform}{\ensuremath{I}\xspace}
\newcommand{\uninform}{\ensuremath{U}\xspace}
\newcommand{\edgeLengths}{\ensuremath{\bm \nu}\xspace}
\newcommand{\patProbSym}{\ensuremath{\mathbb P}\xspace}
\renewcommand{\Pr}{\patProbSym}
\newcommand{\patProb}[5]{\ensuremath{\patProbSym(#1,#2,#3,#4,#5)}\xspace}
\newcommand{\patClassSym}{\ensuremath{\mathcal C}\xspace}
\newcommand{\patClass}[3]{\ensuremath{\patClassSym_{#1,#2,#3}}\xspace}
\newcommand{\stateOf}[1]{\ensuremath{c(#1)}\xspace}

\newcommand{\expectation}{{\mathbb{E}}}
\newcommand{\expect}[1]{\expectation\left[#1\right]}
\newcommand{\pvalue}{$P$-value\xspace}
\newcommand{\pvalues}{$P$-values\xspace}
\newcommand{\myFigLabel}[1]{\stepcounter{myFigCounter} \newcounter{#1} \setcounter{#1}{\value{myFigCounter}} Fig \arabic{#1}}

\newcounter{myFigCounter}
\begin{document}
\section*{Calculating the probability of classes of data pattern}
Given:
\begin{compactitem}
	\item a bifurcating (``binary'') tree, $T$, with root node (vertex) denoted \treeRoot,
	\item with a set of edge lengths, \edgeLengths, defined (not to be estimated), and
	\item  a fully parameterized model of character state change (parameters of the model $\bm\theta$ are fixed),
\end{compactitem}
we would like to be able to calculate the probability of events such as: $\Pr(x\in\patClass{\inform}{\numObserved}{\parsLength}(\treeRoot) \mid T, \edgeLengths, \theta)$ where $x$ refers to a character (a column in a taxa by characters data matrix) and \patClassSym refers to a class of possible data patterns.
The indexing and notation for referring to the class of patterns is as follows:
\begin{compactitem}
	\item argument in parentheses: a node. All subscripts described below refer to properties of the subtree rooted at this node;
	\item first subscript: \inform for parsimony-informative or \uninform for parsimony-uninformative;
	\item second subscript: the number of states observed in tips that are descendants of this node;
	\item third subscript: the parsimony length of the subtree rooted at the node.
\end{compactitem}

Note that the subscripting of \patClassSym ensures that the classes of interest are mutually exclusive.
Thus if traverse the tree in post-order (visiting nodes from tips to root), we can calculate probabilities for each possible class of patterns by summing the probabilities of patterns.
As with Felsenstein's pruning algorithm \citep{Felsenstein1981a}, we must consider each possible state at each ancestral node.
In order to combine partial patterns for child nodes into the appropriate class at their parent's node, we must also maintain the set of states in the subtree.

For each node, $n$, we will calculate and store \patProb{i}{j}{k}{m}{n} where:
\begin{compactitem}
	\item $i$ is \inform  or \uninform,
	\item $j$ is a set of states observed in the tips of the subtree rooted at node $n$,
	\item $k$ is the parsimony length of the subtree rooted at node $n$,
	\item $m$ is a an ancestral character state (a latent variable in this system, because we do not observe these states).
\end{compactitem}
\patProb{i}{j}{k}{m}{n} is defined as
\begin{equation}
	\patProb{i}{j}{k}{m}{n} = \Pr\left(x\in\patClass{i}{|j|}{k}(n) \mid T, \edgeLengths, \theta, \stateOf{n}=m\right)
\end{equation}
Where \stateOf{x} denotes the character state of node $n$.





\section*{{\color{red}Future work - notes below here are not in a finished state}}
\subsection*{Patterns that support a branch.}
Here we consider the following problem: what is the probability of generating a pattern that is shorter on $T_1$ than on $T_{e2}$ or $T_{e3}$ where $T_{e2}$ and $T_{e3}$ are the two distinct tree topologies that are NNI neighbors of tree $T_1$ but do not contain edge $e$.
Such a character would ``support'' edge $e$.
The set of such patterns will be denoted $\mathcal{S}$.

Note that we are not guaranteeting that all (or even that {\em any}) minimal-length trees  for the character contain edge $e$. 
Rather we are assessing a form of local support in which we consider the rest of the tree to be provisionally correct, and are interested in how many characters vary in length across the edge.


Let $\ell_i$ be the length of the character on tree $i$ across the 5 edges of interest (all edges adjacent to the nodes connected by $e$).

Without loss of generality, we will root the tree, $T_1$ across the edge of interest.
The subtrees of $T_1$ will be $((a,b),(c,d))$. The subtrees of $T_2$ will be $((a,c),(b,d))$; The subtrees of $T_3$ will be arranged $((a,d),(b,c))$




Note that $\ell_i \in \{0, 1, 2, 3\}$ because there are three Fitch down-pass nodes considered when sweeping over this part of the tree (thus there can be a maximum of 3 steps added because of the configuration of $a,b,c,d$.
We will prove that, for all patterns in $\mathcal{S}$
\begin{compactenum}
	\item  $\ell_1 \neq 0$  (intersections present between all four subtrees $\rightarrow \ell_i = 1 \forall i$
	\item  $\ell_1 \neq 2$
	\item  $\ell_1 \neq 3$ ( $\ell_2 \leq 3$ and $\ell_3 \leq 3$, so if $\ell_1=3$ then the character cannot be shorter on $T_1$
\end{compactenum}

\bibliography{phylo}
\end{document}

